\documentclass[conference]{IEEEtran}

\usepackage{cite}

\ifCLASSINFOpdf
  \usepackage[pdftex]{graphicx}
  % declare the path(s) where your graphic files are
  %\graphicspath{{../pdf/}{../jpeg/}}
  % and their extensions so you won't have to specify these with
  % every instance of \includegraphics
  \DeclareGraphicsExtensions{.pdf,.jpeg,.png}
\else
  % or other class option (dvipsone, dvipdf, if not using dvips). graphicx
  % will default to the driver specified in the system graphics.cfg if no
  % driver is specified.
  \usepackage[dvips]{graphicx}
  % declare the path(s) where your graphic files are
  % \graphicspath{{../eps/}}
  % and their extensions so you won't have to specify these with
  % every instance of \includegraphics
  \DeclareGraphicsExtensions{.eps}
\fi

\usepackage{amsmath}

\usepackage{algorithmic}

\usepackage{fixltx2e}

\usepackage{url}

\hyphenation{op-tical net-works semi-conduc-tor}


\begin{document}
%
% paper title
% Titles are generally capitalized except for words such as a, an, and, as,
% at, but, by, for, in, nor, of, on, or, the, to and up, which are usually
% not capitalized unless they are the first or last word of the title.
% Linebreaks \\ can be used within to get better formatting as desired.
% Do not put math or special symbols in the title.
\title{Bare Demo of IEEEtran.cls\\ for your report}

% author names and affiliations
% use a multiple column layout for up to three different
% affiliations
\author{\IEEEauthorblockN{Firstname Lastname}
\IEEEauthorblockA{Department of Computer Science\\
University of Helsinki\\
Finland\\
Email: firstname.lastname@cs.helsinki.fi}
}

% make the title area
\maketitle

% As a general rule, do not put math, special symbols or citations
% in the abstract
\begin{abstract}
The abstract goes here.
\end{abstract}


\section{Introduction}
% no \IEEEPARstart
This demo file is intended to serve as a ``starter file''
for your report produced under \LaTeX\ using
IEEEtran.cls version 1.8b and later (Adapted from the IEEE conference bare demo by Michael Shell).
% You must have at least 2 lines in the paragraph with the drop letter
% (should never be an issue)
I wish you the best of success.

\subsection{Subsection Heading Here}
Subsection text here.

\subsubsection{Subsubsection Heading Here}
Subsubsection text here.


\section{Conclusion}
The conclusion goes here.

% conference papers do not normally have an appendix

% references section

% can use a bibliography generated by BibTeX as a .bbl file
% BibTeX documentation can be easily obtained at:
% http://mirror.ctan.org/biblio/bibtex/contrib/doc/
% The IEEEtran BibTeX style support page is at:
% http://www.michaelshell.org/tex/ieeetran/bibtex/
%\bibliographystyle{IEEEtran}
% argument is your BibTeX string definitions and bibliography database(s)
%\bibliography{IEEEabrv,../bib/paper}
%
% <OR> manually copy in the resultant .bbl file
% set second argument of \begin to the number of references
% (used to reserve space for the reference number labels box)
\begin{thebibliography}{1}

\bibitem{IEEEhowto:kopka}
H.~Kopka and P.~W. Daly, \emph{A Guide to \LaTeX}, 3rd~ed.\hskip 1em plus 0.5em minus 0.4em\relax Harlow, England: Addison-Wesley, 1999.

\bibitem{MCC:collier}
S. E. Collier, \emph{The Emerging Enernet: Convergence of the Smart Grid with the Internet of Things}. \hskip 1em plus 0.5em minus 0.4em\relax Abilene, TX: Milsoft Utility Solutions, 2015.

\bibitem{MCC:kovatsch}
M. Kovatsch, S. Mayer and B. Ostermaier, \emph{Moving Application Logic from the Firmware to the Cloud: Towards the Thin Server Architecture for the Internet of Things}. \hskip 1em plus 0.5em minus 0.4em\relax Zurich, Switzerland: Institute for Pervasive Computing, 2012.

\bibitem{MCC:rost}
P. Rost, C. J. Bernardos, A. De Domenico, M. Di Girolamo, M. Lalam, A. Maeder, D. Sabella and D. Wübben, \emph{Cloud Technologies for
Flexible 5G Radio Access Networks}. \hskip 1em plus 0.5em minus 0.4em\relax IEEE Communications Magazine, 2014.

\bibitem{MCC:yin}
Z. Yin, F. R. Yu, S. Bu and Zhu Han, \emph{Joint Cloud and Wireless Networks Operations in Mobile Cloud Computing Environments With Telecom Operator Cloud}. \hskip 1em plus 0.5em minus 0.4em\relax IEEE Transactions on Wireless Communications, 2015.

\bibitem{MCC:miluzzo}
E. Miluzzo, R. Cáceres and Y. Chen, \emph{Vision: mClouds – Computing on Clouds of Mobile Devices}. \hskip 1em plus 0.5em minus 0.4em\relax Florham Park, New Jersey: AT\&T Labs, 2012.

\bibitem{MCC:giurgiu}
I. Giurgiu, \emph{Understanding Performance Modeling for Modular Mobile-Cloud Applications}. \hskip 1em plus 0.5em minus 0.4em\relax ETH Zurich: Dept. of Computer Science, 2012.

\bibitem{MCC:shen}
B. Yin, W. Shen, L. X. Cai and Y. Cheng, \emph{A Mobile Cloud Computing Middleware for Low Latency Offloading of Big Data}. \hskip 1em plus 0.5em minus 0.4em\relax Chicago, Illinois: Illinois Institute of Technology, 2015.

\bibitem{MCC:wang}
Y. Wang, I. Chen and D. Wang, \emph{A Survey of Mobile Cloud Computing Applications: Perspectives and Challenges}. \hskip 1em plus 0.5em minus 0.4em\relax Springer Science, 2014.

\bibitem{MCC:fernando}
N. Fernando, S. W. Loke and W. Rahayu, \emph{Mobile cloud computing: A survey}. \hskip 1em plus 0.5em minus 0.4em\relax Australia: La Trobe University, 2013.

\bibitem{MCC:khan}
A. N. Khan, M. L. Mat Kiah, S. U. Khan and S. A. Madani, \emph{Towards secure mobile cloud computing: A survey}. \hskip 1em plus 0.5em minus 0.4em\relax Future Generation Computer Systems, 2013.

\bibitem{MCC:xu}
X. Xu \emph{From cloud computing to cloud manufacturing}. \hskip 1em plus 0.5em minus 0.4em\relax Auckland, New Zealand: University of Auckland, 2012.

\end{thebibliography}




% that's all folks
\end{document}


