\documentclass[conference]{IEEEtran}

\usepackage{cite}

\ifCLASSINFOpdf
  \usepackage[pdftex]{graphicx}
  \DeclareGraphicsExtensions{.pdf,.jpeg,.png}
\else
  \usepackage[dvips]{graphicx}
  \DeclareGraphicsExtensions{.eps}
\fi

\usepackage{amsmath}

\usepackage{algorithmic}

\usepackage{fixltx2e}

\usepackage{url}

\hyphenation{op-tical net-works semi-conduc-tor}


\begin{document}
\title{Future Radio Access Networks Empowering Mobile Cloud Computing}

\author{\IEEEauthorblockN{Niko Kortstr{\"o}m}
\IEEEauthorblockA{Department of Computer Science\\
University of Helsinki\\
Finland\\
Email: niko.kortstrom@cs.helsinki.fi}
}

\maketitle

\begin{abstract}
This paper begins by introducing mobile cloud computing. This includes general overview, different approaches to, and types of mobile cloud computing. I will also introduce some example use cases. Mobile cloud computing is approached in a way that should make it really easy to grasp the technology's main purposes and techniques.
\par
Next we will dive into the world of mobile networking. Some basic things will be introduced but you will probably need to have a little more background information, than with mobile cloud computing in order to fully understand the details. Basic idea however, should be easy to get into, even without prior experience on the subject.
\par
The relationship of these two technologies will be examined. They are both being developed as we speak to answer the needs of the future. Therefore we are examining their relationship now and what kind the relationship will be in the future.
\par
To put it shortly, the paper offers an overview of both: mobile cloud computing and mobile networking. Then it will continue on to examine their interaction and how do they affect each other. I will also offer some personal opinions how the subject should be approached.
\end{abstract}

\section{Introduction}
The importance of Internet of Things has been underlined a lot pretty much everywhere by experts of different research areas. It is seen as an inevitable direction that technology will aim for in the future. Having computational power available in pretty much everything isn't really useful as such. We need to be able to gather and combine all the data we can get from these different objects. Effective way of issuing commands for many of these objects is also needed. This is why we need to find the best possible way for the smart things to communicate.
\par
Amounts of data gathered from people every day has already reached huge dimensions. With more devices to collect data from their users and also surrounding people, the amount of data collected will obviously increase. We need a secure, reliable and efficient way to transfer this data from the device that collects it to the system that wants to analyze it and then propagate the data forward or use it for various purposes.
\par
There should also be many new and interesting ways to use the mobile data for different things. Besides having more data, it will also be more diverse as it comes from so heterogenous pool of sources. There are already visions for really innovating things to do with this data stream that seems pretty limitless. An example could be smart public transportation systems that book more departures based on how many possible passengers are in the vicinity.
\par
Computational load generated by having to analyze all this data and use it for something, is also massive. We will go through few types of mobile cloud computing that could be used to lift some of the weight from both, network connections and remote data centers. Another one of our topics are the future networking technologies. These will play an important role in enabling the IoT even while new technologies attempt to help them in this task.
\par
The interesting question here is that based on interaction of future mobile networks and mobile cloud computing, should their research consider each other? Should another one's development be based on what the other can achieve or should they both consider other's strengths? Do they have overlapping goals or even some features that wouldn't necessarily have to be implemented in a future scenario where other one of the technologies has already been implemented?

\section{Mobile Cloud Computing}
Mobile devices lack the computational power residing in personal computers. According to Satyanarayanan et al. this will always be the case because improving size, weight and battery life are higher priorities in designing them[5]. This is expected to hold true in the future as well. Of course, while being a design focus, battery life is still an issue when using mobile devices. There are also many other issues like problems with connectivity. Availability of network connection, lower data rates and higher latency are some of these. Most applications can still be ported to run on mobile devices. Many of them are simple enough that the lack of resources is not an issue at all. Sometimes it can also be a good solution to limit the usability of an application to better fit it into its new environment. This can mean offering only a few of the qualities available on the desktop app. An approach like this can be called creating companion apps. For example a huge multiplayer games companion app can simply offer a connection to the in-game chat or your characters inventory.
\par
A hot topic called internet of things can benefit a lot from mobile cloud computing. When computational capacity is embedded into things that are not originally build for it, the computational resources at their disposal are most likely close to non-existent. Using them as mere thin clients is a good option until computational capacity is easy to embed even in the simplest of things. This means that they can just provide inputs to applications running on the cloud. After completing the computation cloud can transmit the result back to same, or some other thing for displayal.
\par
There are numerous problems that can't be solved solely on the limited resources of modern mobile devices. Some of these problems are image processing, natural language processing and multimedia search[1]. This is where mobile cloud computing comes into play. Using a much more powerful device or collection of similar devices devices can help.  Using differect kinds of clouds to carry out the brunt of computing can be used to solve the issue of limited computational resources. In this situation, a mobile device could possibly be used just as a mean to display results of computation. As mentioned, there are several types of clouds that mobile devices can make use of:\\

\subsection{Remote Cloud}
This type of mobile cloud computing is mostly similar with traditional cloud computing. A user offloads some computations to a powerful remote cloud. The difference is that in this case the user interacts with the cloud via a mobile device instead of, for example a laptop or a desktop computer. Everyday services that make use of the remote cloud include some of the most widely used applications and web sites in the world. Some of these are Facebook, Google search and Outlook.

\subsection{Cloudlets}
Cloudlets form a far less known type of mobile cloud computing. Compared to remote clouds, cloudlets can be seen as less powerful but more closely located clouds. Whereas remote clouds can be seen as huge clusters or warehouses full of computational power, cloudlet can be just a single laptop moving with the user or a desktop computer situated at a public location.

\subsection{Cloud of Mobile Devices}
Mobile devices can also be used to form clouds amongst themselves. For this to be possible, users must be willing to submit their mobile device's computational resources to be used by someone else. Another possibility is that multiple users are interested in the result of same computational task. In this case mobile devices will divide the task into smaller pieces and distribute it between the interested mobile devices. The computational capacity of mobile devices is increasing rapidly. Much of this capacity is often not being used. Having a way to harness this unused power could be a great benefit to both, the users offering resources and users making use of these peer-to-peer clouds.\\

\par
Main benefit of the more closely located clouds is not having to use long range connections. There can be many different reasons that it just isn't a valid option at the time to make use of remote data centers. It might be that it is too battery draining, pricey or time consuming to transmit over long ranges. There are also cases that long range connectivity isn't available as depicted in the disaster scenario of Wang et al.[2].
\par
Bringing cloud computing closer to the user has been envisioned to happen via public cloudlets located as densely as Wi-Fi access points in todays world[5]. This would require numerous different parties to want to establish cloudlets. This would mean that they need to gain some sort of benefit out of setting up the infrastructure. Wi-Fi access points are often deployed in public places to attract customers and offer them something to do while waiting. Similar to this case, if applications requiring the use of cloudlets are interesting enough, people could seek out vicinity of cloudslets. If implemented correctly, cloudlets could also offer a much faster, better and as such, more enjoyable experience for already existing applications. This might as well be enough to attract people to use them. Obviously to gain financial benefits, cloudlet owners could also sell their customers' personal information to third parties. Owners of different facilities could also allow third parties to set up their own cloudlets a fee.
\par
Probably the most well known peer-to-peer systems are BitTorrent applications used to distribute entertainment. An application making use of a cloud of mobile devices can also be seen as a kind of peer-to-peer system. BitTorrent systems often encourage users to upload their already downloaded files to other users. Offering them higher download data rates in return is one widely used incentive. An example of a possible incentive[3] in case of mobile computation distribution is for service providers to offer discounts for users who distribute computation in cases of congested network bandwidth.
\par
There are also reasons why users should offload some of their computation when possible, even if they aren't paid for it. Usually offloading computation directly helps with an inherent limiting factor of mobility: battery life. This is quite clear as heavy computation consumes battery fast, minimal computation consumes it a lot slower. However, the issue isn't quite that simple since we also have to account for the amount of battery used by network connection. Connections can limit the possibilities of mobile devices cloud usage if not available but they can also consume a lot of power when used. We must make sure that energy cost of offloading computation isn't too high compared to the energy saved by not executing the computation locally.
\par
Trade-offs of mobile cloud computing must be considered. For this purpose there are a lot of different models[1] to closely look into the cost-benefit relationship of mobile cloud computing. These models can possibly be used by humans to decide if it would be useful to offload some of the computation of a specific application. Since benefit of offloading computation is often environment dependant, it is also useful for the application itself to be able to determine whether to offload and even how to offload.
\par
When discussing moving computation from one entity to another we obviously need to have some kind of method to transfer the data efficiently. Three different methods are considered suitable for this purpose[1]: client-server communication, virtualization and mobile agent. Client-server communication requires the cloud to have some kind of pre-installed interface available for the client to call. When using virtualization, an image of the current memory state of the client is transferred as an image to the server where its execution is resumed. Mobile agent approach includes splitting up the program code from suitable points and sending it ahead to the cloud to be executed. These points can be chosen by the programmer beforehand or the application can for example use an analyzer to find them[2].
\par
Distributing computing to multiple mobile devices also offers other possibilities. One such is also distributing the applications input or sensing capabilities. Distributing a problem to be solved with the help of multiple users is called crowdsourcing. Users can collectively form larger data sets to be processed. Larger data sets often mean more accurate and reliable results, no matter what sort of data we are talking about. There are many[4] possible applications that could benefit greatly from such an approach.
\par
Modern mobile devices are able to collect enormous amounts of information about our habits, schedules and interests etc. We also use mobile devices to access vital services such as banking. Making applications residing in mobile devices interact with the cloud, directly causes additional security threats to both data about our everyday life and different kinds of personal information. Making sure that no data is disclosed to unwanted parties is a requirement that should absolutely be met before making use of cloud in an application. Additional aspect to consider is that in traditional cloud computing, cloud is usually owned and maintained by some well knows corporation that is considered quite trustworthy. In cases of cloudlets and mobile clouds, targets of offloading cannot be always known beforehand. This makes ensuring security and privacy even more difficult.
\par
All these different kinds of clouds, or more precisely combining them, also means rethinking the architecture of applications that we develop. Not only applications might have to be able to choose whether to offload computing, they may also need to pick the best destination for offloading. Having different layers of cloud is a new factor to keep in mind. Programmers need to be able to make applications smart enough for them to get the maximal benefit. This isn't simple as there are many things to compare, like available resources, data rates and network congestion.
\par
Cloudlet and mobile device solutions also contain additional challenges due to mobile devices moving constantly along with their users. Some of the great benefits of these approaches are saving energy and radio access bandwidth by using bluetooth for example. Transmitting the results of offloaded computation while it is still possible, before the device that issued the task moves out of range is essential. How to know make sure this happens is a great challenge for application designers[3]. Losing all the results of offloaded computation right before it's finished can result in much worse processing time, compared to not distributing computing at all. One way to solve this problem is location based offloading[4]. Of course querying users' location adds to the computational needs so it would have to be used sparingly. This is again one problem that could use a powerful artificial intelligence to be solved.

\section{Improving Mobile Networks}
At the moment mobile networks are using the fourth generation techology, LTE. In essence, user equipment (mobile phone etc.) connect to the radio access network. Radio access network is communicating with the mobile core that is connected to the Internet. This is a simplified version of the way our mobile phones get access to the Internet in our everyday life.
\subsection{Requirements of Fifth Generation Mobile Networking}
Fifth generation mobile communication technology is to be released for consumer use in 2020. There are numerous important and really challenging qualities that are required to be met before the next generation cellular communications techology can be released. 5g standards are yet to be created although the work on them should start fairly soon (http://www.ericsson.com/research-blog/lte/release-14-the-start-of-5g-standardization/). Still, many expectations have already been visioned.
\par
Gohil et al. [6] offer their views on what will and should be included in next generation radio access networks. This paper seems like an early review of their opinions rather than evidence based research as to what 5g will really be. It still offers a good glance at the subject and introduces some other related work. Some of the technical qualities highlighted in the paper include seamless cooperation of different radio access technologies, combining cellular and ip connectivity and new protocol stack architecture which all sound interesting and powerful concepts. They also list some practical features associated with 5g. Advanced billing interfaces, large data rates, better consistency and reduced latency are some of these.
\par
Andrews et al. [7] offer similar but more recent and in my opinion much more accurate description of the fifth generation of cellular networking. Requirements set for 5g are described pretty accurately in the paper. They include areas of data rate, latency and energy and cost efficiency. It is also stated that while great challenged to be achieved, all of the goals don't necessarily have to work in practise at the same time. Example is given that streaming high-definition video needs the high data rates but can do with a higher latency. Then again the situation is reversed for driverless cars, low latency is crucial but data rates can be lower. Coming of internet of things also suggests a need for supporting huge amounts of connections from really heterogenous device base. Low rate device-to-device connections functioning side by side with high rate mobile user connections require a new approach to mobile core management and control functions. These are the components responsible for orchestrating data flow and keeping the traffic moving, so to say.
\par
Three clear requirements are set for the data rates needed by ever-advancing technology in the near future [7].
\begin{itemize}
\item Data rates should be increased by approximately thousand times for each area unit.
\item The worst reasonably expectable user data rate should range from 100 Mbps to 1 Gbps. At the moment this value is around 1Mbps so the advance must be at least hundred fold.
\item Best possible rate, while mostly a marketing number, should also be increased greatly. This value is depicted to be around tens of gigabits per second in the near future.
\end{itemize}
\par
Fourth generation radio access networks offer roundtrip latencies of around 15 ms. It is stated that current applications can tolerate this well but some of the real time applications visioned for the era of 5g networks need faster response times. An example of this would be virtual reality interaction that should always fulfill certain rates in order for human mind not to register responses being a little delayed. Roundtrip latency of around 1ms should be delivered to be enough for all the applications envisioned for near future.
\par
Another much discussed subject is energy efficiency. It would obviously be ideal to reduce energy and financial costs but it is stated that this isn't on the top of requirements of 5g. It is, however, required that the costs wouldn't rise. It is stated [7] that environmental, logistical, cost ad battery technology cannot withstand ever-increasing power consumption. In practise this means that as data rates are increased by for example a factor of hundred, energy and monetary cost per bit should be reduced to one percent of the current cost in order to keep the overall cost constant.
\subsection{Reaching for the Goals}
There are many different techniques designed to overcome these challenges. Many promising approaches have already been developed so far. Still, as even the standards are missing, the technologies to be used are far from certain. Still, the development of the techniques is ongoing and as surely experiences with different approaches can help choosing the right standards.
\par
For energy efficiency, different approaches can be combined to achieve the maximal gains. Some of the solutions suggested are: [7].
\begin{itemize}
\item allocating resources in a energy efficient way
\item not having unnecessary basestations
\item running algorithms that turn basestations off when not needed and on again on request
\item using renewable energy to power basestations, for example by attaching solar panels to them
\item energy efficient hardware solutions
\end{itemize}
\par
Some ways for operators to deal with financial costs related to moving into 5g are also presented [7]. Needed additional infrastructure could be constructed by some outside parties. Right to use them for offering connections to end users would then be rented to operators. Increased backhaul requirements gain a lot of help because of overall spreading of fiber connections. Some other ways to help deal with them are better wireless backhaul solutions and optimizing the backhaul operations.
\par
The paper [7] lists ultra-densification, mmWave and massive multiple-input multiple-output (massive MIMO) along with virtualization as the defining factors of future radio access networks. The combination of these approaches is seen as the keys to provide ubiquous, high-rate, low-latency connections to all mobile users.
\par
Densification means simply making cells served by a radio access point smaller. Here cell means an area that a basestation can be responsible for offering connections to. This approach has been used in previous radio access improvements and ultra-densification can be said just to mean taking it to yet another level. The most important benefit of this approach is that spectrum reserved by a user connection via a basestation accommodates a smaller area. In other words, smaller coverage area per basestation means less users connecting via that particular basestation. Obviously this means that there are more basestation resources and more wave frequencies available for each user.
\par
The small cell approach doesn't come without downsides. Some of the associated challenges listed are supporting mobility between cells and increased costs of installing and maintaining all the basestations. Keeping the energy and financial costs low enough for the business to be profitable, small cell devices have to be a lot cheaper and more energy-efficient than today's basestations.
\par
Smaller areas served by a single unit mean more need to transport computing from one basestation to another while the mobile user is on the move. Efficient alghorithms are needed to handle the additional computing caused. Another challenge that cannot be avoided is adding yet another technology into the mix. The fact that the system must be able to support more technologies at the same time increases required complexity of the system.
\par
At the moment wireless communication is mainly propagated via microwave frequencies starting from hundreds of MHz and ranging to few GHz [7]. This isn't too big of a spectrum and it is becoming increasingly congested. Improvements like radio cloud and ultra-densification do help with this issue but don't solve it completely.
\par
The new wave frequencies planned for next generation wireless networking reside at 30 to 300 GHz range. This range is called mmWave because wave length at these frequencies varies from one millimeter to ten. The mmWave frequencies have not been used before because it has been considered unfit for longer range wireless communication. This is due to it being very susceptible to interferance from weather effects and various objects, such as walls, between the transmitter and the receiver.
\par
Rapport et al. [8] did extensive research on mmWave, its viability and effectibility. They conducted important experiments with mmWave both indoors and outdoors. Different kinds of blockages and interferences were also included. Their tests verified that buildings indeed affect the waves a lot as results from Texas were better than those from New York. However with proper network planning and density the mmWave should work just fine for propagating data over longer ranges. Their tests concluded that 200 meter cell radius is enough to achieve good coverage using mmWave frequencies.
\par
Some other challenges related to benefiting from this technology are narrowness of the beams. In addition to coverage challenges, initializing the connection between mobile users and the basestation becomes harder. Building the systems to provide coverage while keeping the cells from interfering with each other is no simple task with beams that can be described to resemble flashlights [7]. The narrowness of the beam also forces the user equipment to possibly scan numerous different directions before locating the the beam to connect to.
\par
In practice MIMO [7] means making use of multiple antennas on both, transmitting and receiving ends. More antennas transmitting and receiving can be used to reach higher rates. Modern technology also allows multiple basestations to cooperate, even allowing them to turn some of the interference into signals containing data. At the moment MIMOs are said to use two to four antennas for transmitting to the mobile device and maximum of eight to transmit within one basestations area.
\par
Envisioned massive MIMOs can consist of hundreds of antennas within one basestation sector. Realizing basestations with these new MIMOs requires a great deal of restructuring the physical architechture of a basestation. Another thing to consider carefully is how to have MIMOs work together with new small cell and mmWave technologies to achieve maximal benefits.
\par
Virtualization has already been somewhat taken into use with mobile core but there is still a lot of work to be done to extend it farther. There are two important terms related to virtualizing the mobile networks: software defined networking and network function virtualization. Implementations before network function virtualization have always closely paired network functions with the hardware. Now the aim is to move much of these functions to be executed in the cloud, offering improved flexibility and management possibilities. This is the essential meaning of network function virtualization. Software defined networking provides the possibility for external applications to modify the network's capabilities, creating intelligent networks.
\par
Lindholm et al. [9] have studied how to implement the virtualization in practise. In other words, they proposed which parts of the LTE mobile network to virtualize should be virtualized and how to possibly change it's overall architecture. Their approach is based on publish-subscribe model where mobile core subscribes to changes in some functions of radio access network and vice versa. This seems like a well suited and efficient way to handle the communication in a modern mobile networking system.
\par
On the hardware level virtualization allows a new design approach, where a traditional radio access point is somewhat divided in two. Remote radio heads handle physically exchanging data with the user equipment. Data centers are used to connect to multiple remote radio heads and handle the processing for them. This is described in more detail for instance by Rost et al. [10].
\par
They view the mobile network stack divided to seven layers:
\begin{itemize}
\item Network management
\item Admission/congestion control
\item Radio resource management
\item Medium access control
\item Physical layer
\item Radio frequency
\end{itemize}
As much as the first six of the are seen to be possible to be virtualized in future implementations. They also refer to flexible functional splitting, meaning that every layer shouldn't just be automatically virtualized. As discussed, virtualization allows creation of more intelligent networks. This is where it can be realized, as we can choose which parts would be useful to be virtualized at runtime. This is yet another gain from the new architecture, in a way making the architecture itself better by adapting it to different situations.
\par
As we have discussed, virtualization is a big part of the ongoing mobile network development. Most have probably heard of system architectures consisting of different services, such as infrastructure as a service, platform as a service and software as a service. Nikaein et al. [11] go as far as to introduce a similar consept related to mobile networks, radio access network as a service. This is a term that to me really demonstrates the possible gains from this new approach. The amount of flexibility and managebility gained seems to get new proportions when approached like this.
\par
In addition to the new qualities introduced to improve mobile networks, there are also possibilities to improve the existing ones once the architectural changes have been implemented. One such approach is improving the caching performed by radio access networks [12]. Ao et al. introduce their views on how to increase benefits from caching with the help of small cells.
\par
Multiple radio access points covering an area mean multiple caches available to deliver cached data quickly to the receiving users. They give an example of caching video files based on their popularity (the most popular video being cached to most radio access points). Most downloaded files being cached most widely should lead to high cache hit rates. This way some of the ever-increasing data download can be handled even more efficiently. This approach could possibly reduce the stress from core network by a lot.

\section{Mobile Cloud Computing in the Context of Improved Cellular Networks}
So how does this all pan out? How do the improved mobile networks affect existing and new mobile cloud computing applications? Is there some considerations to be acquired from the field of mobile cloud computing for designing the next generation networks? This is of course just one technological relationship but in my opinion it's important to discuss the possible interactions of the two topics.
\par
First I'd like to discuss the cloudlet approach to mobile cloud computing. I somewhat question the need for public cloudlets for computation offloading. It is definately true that computational needs are exceeding the capabilities of mobile devices. However I would argue that with the coming of 5g networking many of the benefits of cloudlets can be answered with traditional cloud computing. Data rates far exceeding the current ones and a lot lower latencies should prove really powerful tools for offloading computation. This could hold even while internet of things is mixing up things and possibly causing congestion in the network.
\par
Reduced latencies and increased data rates make transmitting virtual machines or remote calling functions from longs distances even more viable. Perhaph viable enough to answer the needs of present and even near future applications. The challenges of the cloudlet approach, building the infrastructure and modifying application architectures to be able to use cloudlets effectively are no small tasks. In addition, the computational power of huge datacenter far exceeds that of cloudlets. It might be a lot more practical and even computationally effective to stick with offloading to remote cloud instead of opting for this new approach.
\par
Standardization for 5g is most likely based on exactly the requirements visioned for future needs of mobile computing that the cloudlet and mobile cloud approaches are also trying to solve. If research indeed is successful in meeting the set goals, it would be intuitive to think that there is no real need for other solutions.
\par
Having no need for widespread implementation of the newer mobile cloud solutions still doesn't mean that there wouldn't be specific scenarios where they would be of great benefit. Local proximity is already an important point of many applications (for example some dating applications). Taking this a step further and opting only for close range connections could be valid selling point. Some cloudlet applications could also be visioned, perhaps even own applications set up in public cloudlets. This could mean having to join others in a specific public location to collaborate in whatever is the goal of said application. This would introduce yet another social factor to world of mobile computing.
\par
One reason to offload where ever possible is saving money. As discussed before, avoiding roaming costs or even saving on data transfers when you are paying by usage are valid reasons. Still, widespread free wi-fi connections offer possibilities for traditional cloud usage, albeit data rates are often less than optimal.
\par
Another aspect speaking for close range computation offloading is saving battery.  Closer range connection techniques do require less battery and in this way it would be really useful to be able to offload close whenever you can. It can however, be hard to determine if benefits overcome the costs in case of cloudlets.
\par
In my opinion mobile clouds offer a more viable option compared to cloudlets. The thing is, when the computation is offloaded to other mobile devices, the infrastructure already exists. It should also be quite clear that there are a lot of spare resources available most of the time. For example my own phone sits pretty much idle eight hours a day when working There is no question that a way to harness these resources would benfit all parties.
\par
Of course a lot of work needs to be done to design applications that can make use of other mobile devices' resources. Also is is obvious that mobile devices don't possess even the computational powers that cloudlets do. Their sheer numbers might however prove to offer similar or even better possibilities, should people opt to participate in enabling this approach. After all, applications using this approach must be really widely adopted for it to be useful. Otherwise what are the chances that there is for example another user in your bluetooth range when you need him/her?
\par
Peer-to-peer applications are really widely used and there is also extensive research on the subject. This should prove to be of great help in implementing clouds of mobile devices. At least to me, they seem like fields that share many similarities. Already existing research can offer good guidelines to start designing your application from or you could possibly even reuse some protocols.
\par
The previously discussed disaster scenario [7] is an obvious example of a case where the mobile network infrastructure is of no use. This is where cloud of mobile device gets an opportunity to shine. As cloudlets could be of great effect as emergency personnel could move them around with them, mobile networks could also be set up quite quickly.
\par
Mobile networks are gaining a lot of flexibility via virtualization thus, it could even be viable to set up remote radio heads after disasters. Having most of the network functions already operational in a data center and just installing the radio frequency transmitters/receivers could possibly be done fairly quickly. Obviously it would not be as quick a way to respond as having mobile devices form their own cloud, but it's something to consider.
\par
Security issues are probably higher when offloading to peers and cloudlets set up by non-experts instead of well established and widely known cloud providers. They might not be high profile targets and so don't face as many threats but the challenge of authenticating the used cloud provider becomes a lot more challenging. You often use the same cloud services you have known for years, but what if you would have to use tens of new cloud providers per day? This could easily lead to people losing some of their data or just refusing to use some applications because of fear, thus limiting their spread.
\par
Challenges related to near-constant mobility of the mobile users are also a challenge that is already being solved in cellular networks. In case of mobile clouds it would have to be figured out all over again. It would probably also be a lot harder, as there might not be similar high level management layers available. Even if some management services were taken in use, they would most likely reside in remote cloud, leading partially back to a traditional approach.
\par
From the mobile data analytics point of view virtualized networking also offers various new possibilities. For example Nokia Liquid Applications (http://networks.nokia.com/portfolio/liquid-net/intelligent-broadband-management/liquid-applications) allows different programs to be run in different parts of the telecommunications infrastructure. This offers operators endless possibilities to handle data in different ways and help make their customers' life better. An example given was instantly notifying drivers on a road of upcoming crash sites etc.
\par
The whole coming of internet of things no doubt will use many short range connections to function. I would imagine many of these will be to the owners cell phone, from which they will be able to access the internet. However, when IoT starts realizing it's full potentials most of the devices are probably outside of teathering range. I think it would also be quite hard to equip each object with it's own connection to mobile network. In my opinion this is one great way to benefit from mobile cloud, offering connectivity to devices that don't possess the long range capabilities themselves.

\section{Conclusion}
Based on looking into both of these subjects: mobile cloud computing and mobile networking, I would say that their goals seem pretty similar. Both are being designer to handle some of the challenges of internet of things and vastly increasing amounts of mobile data. The way I see it, 5g technologies are a necessity and they will be developed no matter what. Usage of mobile cloud computing is in the hands of application developers and as such not destined to spread as widely. They should carefully think of the work put into mobile cloud computing implementations if 5g indeed manages to reach it's goals. This might make some parts of well developed mobile cloud computing systems needless.
\par
It should also be noted that the internet of things will be in wide use only in the future, when 5g networks are already widely available or at least quite close to it. Of course there are many differentiating challenges that mobile clouds are trying to solve, but maybe it should focus precisely on the ones that aren't already being solved by better networks. It could even get achieve better results in the needed areas if it's full focus was on them.
\par
As far as I see it, mobile cloud computing is being researched in the context on today's world. It is however, often pictured as a technology needed to answer to future needs. This is a context where the network connections have already evolved and this should be taken into consideration.
\par
I think more accurate research should be done on the relationship of 5g and mobile cloud computing. Their timelines should be compared and it should be determined how they can benefit from each other. They clearly have similar goals and the research focus should be aimed at the problems that need solving the most.
\par
It might also be that the future computing is so demanding that both of these technologies are required to have a chance at overcoming all the challenges. This is also why their combined effectiveness should be investigated, already in the early stages.
\par
I would also like to note that from the point of view of mobile data analysis, implementing functions in different parts of mobile network via virtualization seems like a topic that offers really great possibilities. There are already innovations being made based on this and probably much more will come up. This is a way to get the analysing funtions truly close to the data to be able to offer truly great response times. This is an example of an area where the two discussed technologies could achieve much by using each other's strengths to deliver minimal latencies where it is needed.

\begin{thebibliography}{1}

\bibitem{MCC:fernando}
N. Fernando, S. W. Loke and W. Rahayu, \emph{Mobile cloud computing: A survey}. \hskip 1em plus 0.5em minus 0.4em\relax Future Generation Computer Systems Volume 29 Issue 1, January 2013.

\bibitem{MCC:wang}
Y. Wang, I. Chen and D. Wang, \emph{A Survey of Mobile Cloud Computing Applications: Perspectives and Challenges}. \hskip 1em plus 0.5em minus 0.4em\relax Wireless Personal Communications: An International Journal Volume 80 Issue 4, February 2015.

\bibitem{MCC:miluzzo}
M. Miluzzo, R. Cáceres and Y. Chen, \emph{Vision: mClouds – Computing on Clouds of Mobile Devices}. \hskip 1em plus 0.5em minus 0.4em\relax MCS '12 Proceedings of the third ACM workshop on Mobile cloud computing and services, 2012.

\bibitem{MCC:chatzimilioudis}
G. Chatzimilioudis, A. Konstantinidis, C. Laoudias and D. Zeinalipour-Yazti, \emph{Crowdsourcing with Smartphones}. \hskip 1em plus 0.5em minus 0.4em\relax IEEE Internet Computing Volume 16 Issue 5, September 2012.

\bibitem{MCC:satyanarayanan}
M. Satyanarayanan, P. Bahl, R. Caceres and N. Davies, \emph{The Case for VM-Based Cloudlets in Mobile Computing}. \hskip 1em plus 0.5em minus 0.4em\relax IEEE Pervasive Computing Volume 8 Issue 4, October 2009 

\bibitem{MCC:gohil}
A. Gohil, H. Modi and S. K. Patel, \emph{5G Technology of Mobile Communication: A Survey}. \hskip 1em plus 0.5em minus 0.4em\relax Intelligent Systems and Signal Processing (ISSP) International Conference, 2013

\bibitem{MCC:andrews}
J. G. Andrews, S. Buzzi, W. Choi, S. V. Hanly, A. Lozano, A. C. K. Soong and J. C. Zhang, \emph{What Will 5G Be?}. \hskip 1em plus 0.5em minus 0.4em\relax IEEE Journal on Selected Areas in Communications Volume 32 Issue 6, June 2014

\bibitem{MCC:rappaport}
T. S. Rappaport, S. Sun, R. Mayzus, H. Zhao, Y. Azar, K. Wang, G. N. Wong, J. K. Schulz, M. Samimi and F. Gutierrez, \emph{Millimeter Wave Mobile Communications for 5G Cellular: It Will Work!}. \hskip 1em plus 0.5em minus 0.4em\relax IEEE Access Volume 1, May 2013 

\bibitem{MCC:lindholm}
H. Lindholm, L. Osmani, H. Flinck, S. Tarkoma and A. Rao, \emph{State Space Analysis to Refactor the Mobile Core}. \hskip 1em plus 0.5em minus 0.4em\relax AllThingsCellular '15 Proceedings of the 5th Workshop on All Things Cellular: Operations, Applications and Challenges, August 2015

\bibitem{MCC:rost}
P. Rost, C. J. Bernardos, A. De Domenico, M. Di Girolamo, M. Lalam, A. Maeder, D. Sabella and D. Wübben, \emph{Cloud Technologies for Flexible 5G Radio Access Networks}. \hskip 1em plus 0.5em minus 0.4em\relax IEEE Communications Magazine Volume 52 Issue 5, May 2014

\bibitem{MCC:nikaein}
N. Nikaein, R. Knopp, L. Gauthier, E. Schiller, T. Braun, D. Pichon, C. Bonnet, F. Kaltenberger and D. Nussbaum, \emph{Demo – Closer to Cloud-RAN: RAN as a Service}. \hskip 1em plus 0.5em minus 0.4em\relax MobiCom '15 Proceedings of the 21st Annual International Conference on Mobile Computing and Networking, September 2015

\bibitem{MCC:ao}
W. Ao and K. Psounis, \emph{Distributed Caching and Small Cell Cooperation for Fast Content Delivery}. \hskip 1em plus 0.5em minus 0.4em\relax MobiHoc '15 Proceedings of the 16th ACM International Symposium on Mobile Ad Hoc Networking and Computing, June 2015

\end{thebibliography}

\end{document}
