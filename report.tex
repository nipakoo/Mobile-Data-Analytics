\documentclass[conference]{IEEEtran}

\usepackage{cite}

\ifCLASSINFOpdf
  \usepackage[pdftex]{graphicx}
  \DeclareGraphicsExtensions{.pdf,.jpeg,.png}
\else
  \usepackage[dvips]{graphicx}
  \DeclareGraphicsExtensions{.eps}
\fi

\usepackage{amsmath}

\usepackage{algorithmic}

\usepackage{fixltx2e}

\usepackage{url}

\hyphenation{op-tical net-works semi-conduc-tor}


\begin{document}
\title{Future Radio Access Networks Empowering Mobile Cloud Computing}

\author{\IEEEauthorblockN{Niko Kortstr{\"o}m}
\IEEEauthorblockA{Department of Computer Science\\
University of Helsinki\\
Finland\\
Email: niko.kortstrom@cs.helsinki.fi}
}

\maketitle

\begin{abstract}
The abstract goes here.
\end{abstract}

\section{Introduction}
Introduction goes here.

\section{Mobile Cloud Computing}
Mobile devices lack the computational power residing in personal computers. According to Satyanarayanan et al. this will always be the case because improving size, weight and battery life are higher priorities in designing them[5]. Of course while a design focus, battery life is still an issue when using mobile devices. There are also many other issues like problems with connectivity. Most applications can still be ported to run on mobile devices. Many of them are simple enough that the lack of resources is not an issue at all. Sometimes it can also be a good solution to limit the usability of an application to better fit it into its new environment. This can mean offering only some qualities of the desktop app. This kind of applications can be called companion apps. For example a huge multiplayer games companion app can simply offer a connection to the in-game chat or your characters inventory.
\par
A hot topic called internet of things can benefit a lot from mobile cloud computing. When computational capacity is embedded into things not originally build for it, their resources are most likely close to non-existent. Using them as mere thin clients, providing inputs to applications running on the cloud is a good option until computational resources of these things evolves.
\par
There are numerous problems that can't be solved solely on the limited resources of modern mobile devices. Some of these problems are image processing, natural language processing and multimedia search[1]. This is where mobile cloud computing comes into play. Using a much more powerful device or collection of devices, in other words a cloud, to carry out the brunt of computing solves this issue. In this situation, a mobile device could possibly be used just as a mean to display results of computation. There are several types of clouds that mobile devices can make use of.\\

\subsubsection{Remote Cloud}
This type of mobile cloud computing is mostly similar with traditional cloud computing. A user offloads some computations to a powerful remote cloud. The difference is that in this case the user interacts with the cloud via a mobile device instead of, for example a laptop or a desktop computer. Everyday services that make use of the remote cloud include some of the most widely used applications and web sites in the world, such as Facebook, Google search and Outlook.

\subsubsection{Cloudlets}
Cloudlets form a far less known type of mobile cloud computing. Compared to remote clouds, cloudlets can be seen as less powerful but more closely located clouds. Whereas remote clouds can be seen as huge clusters or warehouses full of computational power, cloudlet can be just a single laptop moving with the user or a desktop computer situated at a public location.

\subsubsection{Cloud of Mobile Devices}
Mobile devices can also be used to form clouds amongst themselves. For this to be possible, users must be willing to submit their mobile device's computational resources to be used by someone else. Another possibility is that multiple users are interested in the result of same computational task. In this case mobile devices will divide the task into smaller pieces and distribute it between the interested mobile devices. The computational capacity of mobile devices is increasing rapidly. Much of this capacity is often not being used. Having a way to harness this unused power could be a great benefit to both, the users offering and making use of it.\\

\par
Main benefit of the more closely located clouds is not having to use long range connections. There can be many different reasons that it just isn't a valid option at the time to make use of remote data centers. It might be that it is too battery draining, pricey or time consuming to transmit over long ranges. There are also cases that long range connectivity isn't available as depicted in the disaster scenario of Wang et al.[2].
\par
Bringing cloud computing closer to the user has been envisioned to happen via public cloudlets located as densely as Wi-Fi access points in todays world[5]. This would require numerous different parties to want to establish cloudlets, meaning that they need to gain some sort of benefit out of this. Wi-Fi access points are often deployed in public places to attract customers and offer them something to do while waiting for example. Similar to this case, if applications making use of cloudlets are interesting enough, people could seek out vicinity of cloudslets. Obviously to gain financial benefits, cloudlet owners could also sell their customers' personal information to third parties or owners of different facilities could allow third parties to set up their own cloudlets a fee.
\par
Probably the most well known peer-to-peer systems are BitTorrent applications used to distribute entertainment. An application making use of a cloud of mobile devices can also be seen as a kind of peer-to-peer system. BitTorrent systems often encourage users to upload their already downloaded files to other users. Offering them higher download data rates in return is one widely used incentive. An example of a possible incentive[3] in case of mobile computation distribution is for service providers to offer discounts for users who distribute computation in cases of congested network bandwidth.
\par
There are also reasons why users should offload some of their computation when possible, even if they aren't paid for it. Usually offloading computation directly helps with an inherent limiting factor of mobility: battery life. This is quite clear as heavy computation consumes battery fast, minimal computation consumes it a lot slower. However, the issue isn't quite as simple since we also have to account for the amount of battery used by network connection. Connections can limit the possibilities of mobile devices cloud usage if not available but they can also consume a lot of power when used. We must make sure that energy cost of offloading computation isn't too high compared to the energy save.
\par
Trade-offs of mobile cloud computing must be considered. For this purpose there are a lot of different models[1] to closely look into the cost-benefit relationship of mobile cloud computing. These models can possibly be used by humans to deside if it would be useful to offload some of the computation of a specific application. Since benefit of offloading computation is often environment dependant, it is also useful for the application itself to be able to determine whether and how to offload.
\par
When discussing moving computation from one entity to another we obviously need to have some kind of method to transfer the data efficiently. Three different methods are considered suitable for this purpose[1]: client-server communication, virtualization and mobile agent. Client-server communication requires the cloud to have some kind of pre-installed interface available for the client to call. When using virtualization, an image of the current memory state of the client is transferred as an image to the server where its execution is resumed. Mobile agent approach includes splitting up the program code from suitable points and sending it ahead to the cloud to be executed. These points can be chosen by the programmer beforehand or the application can for example use an analyzer to find them[2].
\par
Distributing computing to multiple mobile devices also offers other possibilities. One such is also distributing the applications input or sensing capabilities. Distributing a problem to be solved with the help of multiple users is called crowdsourcing. Users can collectively form larger data sets to be processed. Larger data sets often mean more accurate and reliable results, no matter what data we are talking about. There are many[4] possible applications that could benefit greatly from such an approach.
\par
Modern mobile devices are able to collect enormous amounts of information about our habits, schedules and interests etc. We also use mobile devices to access vital services such as banking. Making applications residing in mobile devices interact with the cloud, directly causes additional security threats to both data about our everyday life and different kinds of personal information. Making sure that no data is disclosed to unwanted parties is a requirement that should absolutely be met before making use of cloud in an application. Additional aspect to consider is that in traditional cloud computing, cloud is usually owned and maintained by some well knows corporation that is considered quite trustworthy. In cases of cloudlets and mobile clouds, targets of offloading cannot be always known beforehand. This makes ensuring security and privacy even more difficult.
\par
All these different kinds of clouds, or more precisely combining them, also means rethinking the architecture of applications that we develop. Not only applications might have to be able to choose whether to offload computing, they may also need to pick the best destination for offloading. Having different layers of cloud is a new factor to keep in mind. Programmers need to be able to make applications smart enough for them to get the maximal benefit. This isn't simple as there are many things to compare, like available resources, data rates and network congestion.
\par
Cloudlet and mobile device solutions also contain additional challenges due to mobile devices moving constantly along with their users. Some of the great benefits of these approaches are saving energy and radio access bandwidth by using bluetooth for example. Transmitting the results of offloaded computation while it is still possible, before the device that issued the task moves out of range is essential. How to know make sure this happens is a great challenge for application designers[3]. Losing all the results of offloaded computation right before it's finished can result in much worse processing time, compared to not distributing computing at all. One way to solve this problem is location based offloading[4]. Of course querying users' location adds to the computational needs so it would have to be used sparingly. This is again one problem that could use a powerful artificial intelligence to be solved.

\section{Radio Access Network Improvements}
Section text here.

\subsection{Decreased Latency}
Subsubsection text here.

\subsection{Increased Data Rates}
Subsubsection text here.

\subsection{Cloud Radio Access Network}
Subsubsection text here.

\section{Effects of Improved Radio Access Networks}
Section text here.

\section{Conclusion}
The conclusion goes here.

\begin{thebibliography}{1}

\bibitem{MCC:fernando}
N. Fernando, S. W. Loke and W. Rahayu, \emph{Mobile cloud computing: A survey}. \hskip 1em plus 0.5em minus 0.4em\relax Future Generation Computer Systems Volume 29 Issue 1, January 2013.

\bibitem{MCC:wang}
Y. Wang, I. Chen and D. Wang, \emph{A Survey of Mobile Cloud Computing Applications: Perspectives and Challenges}. \hskip 1em plus 0.5em minus 0.4em\relax Wireless Personal Communications: An International Journal Volume 80 Issue 4, February 2015.

\bibitem{MCC:miluzzo}
M. Miluzzo, R. Cáceres and Y. Chen, \emph{Vision: mClouds – Computing on Clouds of Mobile Devices} . \hskip 1em plus 0.5em minus 0.4em\relax MCS '12 Proceedings of the third ACM workshop on Mobile cloud computing and services, 2012.

\bibitem{MCC:chatzimilioudis}
G. Chatzimilioudis, A. Konstantinidis, C. Laoudias and D. Zeinalipour-Yazti, \emph{Crowdsourcing with Smartphones}. \hskip 1em plus 0.5em minus 0.4em\relax IEEE Internet Computing Volume 16 Issue 5, September 2012.

\bibitem{MCC:satyanarayanan}
M. Satyanarayanan, P. Bahl, R. Caceres and N. Davies, \emph{The Case for VM-Based Cloudlets in Mobile Computing}. \hskip 1em plus 0.5em minus 0.4em\relax IEEE Pervasive Computing Volume 8 Issue 4, October 2009 

\end{thebibliography}

\end{document}
